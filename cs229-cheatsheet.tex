\documentclass[10pt,landscape]{article}
\usepackage{multicol}
\usepackage{calc}
\usepackage{ifthen}
\usepackage[landscape]{geometry}
\usepackage{amsmath,amsthm,amsfonts,amssymb}
\usepackage{color,graphicx,overpic}
\usepackage{hyperref}
\everymath{\displaystyle}



\pdfinfo{
  /Title (example.pdf)
  /Creator (TeX)
  /Producer (pdfTeX 1.40.0)
  /Author (Seamus)
  /Subject (Example)
  /Keywords (pdflatex, latex,pdftex,tex)}

% This sets page margins to .5 inch if using letter paper, and to 1cm
% if using A4 paper. (This probably isn't strictly necessary.)
% If using another size paper, use default 1cm margins.
\ifthenelse{\lengthtest { \paperwidth = 11in}}
    { \geometry{top=.5in,left=.5in,right=.5in,bottom=.5in} }
    {\ifthenelse{ \lengthtest{ \paperwidth = 297mm}}
        {\geometry{top=1cm,left=1cm,right=1cm,bottom=1cm} }
        {\geometry{top=1cm,left=1cm,right=1cm,bottom=1cm} }
    }

% Turn off header and footer
\pagestyle{empty}

% Redefine section commands to use less space
\makeatletter
\renewcommand{\section}{\@startsection{section}{1}{0mm}%
                                {-1ex plus -.5ex minus -.2ex}%
                                {0.5ex plus .2ex}%x
                                {\normalfont\large\bfseries}}
\renewcommand{\subsection}{\@startsection{subsection}{2}{0mm}%
                                {-1explus -.5ex minus -.2ex}%
                                {0.5ex plus .2ex}%
                                {\normalfont\normalsize\bfseries}}
\renewcommand{\subsubsection}{\@startsection{subsubsection}{3}{0mm}%
                                {-1ex plus -.5ex minus -.2ex}%
                                {1ex plus .2ex}%
                                {\normalfont\small\bfseries}}
\makeatother

% Define BibTeX command
\def\BibTeX{{\rm B\kern-.05em{\sc i\kern-.025em b}\kern-.08em
    T\kern-.1667em\lower.7ex\hbox{E}\kern-.125emX}}

% Don't print section numbers
\setcounter{secnumdepth}{0}


\setlength{\parindent}{0pt}
\setlength{\parskip}{0pt plus 0.5ex}

%My Environments
\newtheorem{example}[section]{Example}
% -----------------------------------------------------------------------

\begin{document}
\raggedright
\footnotesize
\begin{multicols}{3}


% multicol parameters
% These lengths are set only within the two main columns
%\setlength{\columnseprule}{0.25pt}
\setlength{\premulticols}{1pt}
\setlength{\postmulticols}{1pt}
\setlength{\multicolsep}{1pt}
\setlength{\columnsep}{2pt}

\begin{center}
     \Large{\underline{CS229-Cheatsheet}} \\
\end{center}

\section{Supervised Learning}
\begin{itemize}
\item \textbf{Gradient Descent:} to minimize $J(\theta)$, we perform $\theta_j := \theta_j	 - \alpha \frac{\partial}{\partial \theta} J(\theta)$
\item $\nabla_A AB = B^T$ , $\nabla_{A^T} f(A) = (\nabla_A f(A))^T$, $\nabla_A tr ABA^TC = CAB + C^TAB^T$, $\nabla_A |A| = |A| (A^{-1})^T$

\item \textbf{Normal Equations and Least Squares} $J(\theta) = \frac{1}{2}\sum_{i = 1}^{m}(h_\theta(x^i)- y^i)^2 \rightarrow \nabla_\theta J(\theta)  = \nabla_\theta \frac{1}{2}(X\theta -y)^T(X\theta -y) =  X^TX\theta - X^Ty  = 0 \rightarrow X^TX\theta = X^Ty \rightarrow \theta = (X^TX)^{-1}X^Ty$. 
\item \textbf{Locally Weighted Regression} Fit $\theta$ to minimize $\sum_{i = 0}^{m	} (y^i-\theta^Tx^i)^2$ where $w^i = e^{-\frac{(x^i-x)^2}{2\tau^2}}$
\item \textbf{Logistic Regression:} $h_\theta(x) = g(\theta^T x) = \frac{1}{1+ e^{-\theta^Tx}}$, $g(z) = \frac{1}{1+e^{-z}}$, $g'(z) = \frac{d}{dz}\frac{1}{1+e^{-z}} = g(z)(1-g(z))$, $p(y |x; \theta) = (h_\theta(x))^y(1-h_\theta(x))^{1-y}$. $l(\theta) = log L(\theta) = \sum_{i = 1}^{m} y^i logh(x^i) + (1-y^i) log(1-h(x^i))$, $\frac{\partial}{\partial \theta_j} l(\theta) = (y- h_\theta(x))x_j$
\item \textbf{Perceptron Learning Algorithm} $\theta_j := \theta_j + \alpha (y^i - h_\theta(x^i))x_j^i$
\item \textbf{Newton's Method: } $\theta := \theta - \frac{f(\theta)}{f'(\theta)}$, we want the first derivative to be zero, then $\theta := \theta - \frac{l'(\theta)}{l''(\theta)}$, if $\theta$ is a vector then $\theta: = \theta - H^{-1}\nabla_\theta l(\theta)$ where $H_{ij} = \frac{\partial ^2 l(\theta}{\partial \theta_i \partial \theta_j}$
\item \textbf{Exponential Family}  $p(y; \eta)  =b(y)exp(\eta^TT(y) - a(\eta))$
\item \textbf{General Linear Model Assumptions:} 1. $y|x; \eta \sim ExponentialFamily(\eta)$. 2. Given $x$ our goal is to predict the expected value of $T(y)$ which is usually just $y$, so we would like our hypothesis to satisfy $h(x) = E(y|x)$. 3. The natural parameter $\eta$ and inputs $x$ are related linearly. $\eta = \theta^Tx$. 
\item \textbf{Canonical response function:} the distribution's mean as a function of the natural parameter $g(\eta) = E(T(y); \eta)$. 
\end{itemize}
\section*{Generative Learning Algorithm}
\subsection*{Gaussian Discriminant Analysis}
\begin{itemize}
\item $y \sim Bernoulli(\phi), x|y = 0 \sim N(\mu_0, \Sigma), x|y = 1 \sim N(\mu_1, \Sigma)$. 
\item $p(y) = \phi^y (1-\phi)^{1-y}$
\item $p(x|y=0) = \frac{1}{(2\pi)^{n/2} |\Sigma|^{1/2}} exp(-\frac{1}{2}(x-\mu_0)^T\Sigma^{-1}(x-\mu_0))$
\item $p(x|y=1) = \frac{1}{(2\pi)^{n/2} |\Sigma|^{1/2}} exp(-\frac{1}{2}(x-\mu_1)^T\Sigma^{-1}(x-\mu_1))$
\item $l(\phi, \mu_0, \mu_1, \Sigma) = log \prod_{i=1}^m p(x^i, y^i; \phi, \mu_0, \mu_1, \Sigma) = log \prod_{i=1}^m p(x^i| y^i; \phi, \mu_0, \mu_1, \Sigma) p(y^i, \phi)$. 
\item By maximizing $l$ with respect to the parameters, we find the maximum likelihood of the parameters to be:\\
$\phi= \frac{1}{m}1\{y^i = 1\}$\\
$\mu_0 = \frac{\sum_{i = 1}^{m}\{y^i = 0\}x^i}{\sum_{i = 1}^{m}\{y^i = 0\}}$\\
$\mu_1 = \frac{\sum_{i = 1}^{m}\{y^i = 1\}x^i}{\sum_{i = 1}^{m}\{y^i = 1\}}$\\
$\Sigma = \frac{1}{m}\sum_{i = 1}^{m} (x^i - \mu_{y^i})^T (x^i - \mu_{y^i})$
\end{itemize}
\subsection*{Naive Bayes}
\begin{itemize}
\item \textbf{Naive Assumption:} $p(x_1, x_2, \dots |y) = p(x_1|y)p(x_2|y) \dots. = \prod_{i = 1}^n p(x_i|y)$
\end{itemize}
% You can even have references
\rule{0.3\linewidth}{0.25pt}
\scriptsize
\bibliographystyle{abstract}
\bibliography{refFile}
\end{multicols}
\end{document}